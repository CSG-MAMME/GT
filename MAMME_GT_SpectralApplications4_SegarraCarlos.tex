\documentclass[a4paper, 10pt]{article}
\usepackage[utf8]{inputenc} % Change according your file encoding
\usepackage{graphicx}
\usepackage{url}
\usepackage{amsmath}
\usepackage{amsthm}
\usepackage{tikz}
\usepackage{setspace}
\usepackage[a4paper, left=2cm, right=2cm, top=2cm, bottom=2cm]{geometry}

\newtheorem{obs}{Observation}
\newtheorem{theorem}{Theorem}
\newtheorem{lemma}{Lemma}

\theoremstyle{definition} % amsthm only
\newtheorem{definition}{Definition}

%\newtheorem{def}{Definition}

\begin{document}
\onehalfspacing

Carlos Segarra \hfill Tuesday, November 26th

\vspace{15pt}

\textbf{\Large Graph Theory: Some Applications of the Laplacian Matrix - Problem 4}

\vspace{20pt}

\textbf{\textit{(4) A multigraph $G$ is a graph where multiple edges and loops are allowed. Let G be a loopless multigraph. Prove a version of the Matrix-Tree theorem for multigraphs.}}

\vspace{3pt}

Let us recall that a loopless multigraph only differs from a simple graph in the fact that some edges may appear "repeated".
Using this intuition we can think on how the solution of the problem may look like.
To do so, let us use a very simple combinatorial argument.
Given that the we know the number of spanning trees for a simple connected graph $G$, if we now allow multiple edges between two vertices, the number of new spanning trees derived from that original one will be exactly the productory of the number of repeated (or multiple) edges.

To prove the theorem in a more formal manner, we will use this same intution and make use of the extended Cauchy-Binet formula given in class.
Let us state both.

\begin{theorem}[Matrix-Tree Theorem]
    The number $\tau(G)$ of spanning trees of a graph $G$ is
    $$\tau(G) = \text{det}\left(L_{(i,i)}\right) = \frac{1}{n} \mu_2 \dots \mu_n$$
    where $L_{(i,i)}$ denotes the minor of the Laplacian matrix $L$ of $G$ obtained by deleting row $i$ and column $i$, and $0 = \mu_1 < \mu_2 \leq \dots \leq \mu_n$ denote the eigenvalues of $L$.
\end{theorem}

\begin{theorem}[Cauchy-Binet Formula]
    Let $A$ be an $n \times m$ matrix, $B$ an $m \times n$ matrix, and $D$ an $m \times m$ diagonal matrix. Let $[m] := {1, \dots, m}$. For $S \in {[m] \choose n}$, $A_{[n],S}$ the $n \times n$ matrix whose columns are the columns of $A$ at indices from $S$. Then,
    $$\det(ADB) = \sum_{S \in {[m] \choose n}} \det\left(A_{n], S}\right) \det\left(B_{S,[n]}\right) \prod_{i \in S} d_i$$
\end{theorem}

To apply it to multigraphs, we first need to define a version of the laplacian matrix for multigraph.
We use the natural extension of the diagonal and adjacency matrix for multigraphs, where the latter is defined such that $a_{ij} = \left| e_{ij} \in E(G) \right|$, assuming, of course, that we somehow can differentiate them.
Then $\bar{L} = \text{diag}(d_1, \dots, d_n) - A$ accounting for repetitions.

There remains just one key observation to solve this problem.
\begin{lemma}
    Let $L = Q \cdot Q^T$ be the standard decomposition in signed matrices of the laplacian matrix of the graph \emph{removing} repeated edges.
    Let $m$ be the number of edges without repetitions and $k_i, i \in [m]$ the multiplicity of edge $i$.
    Lastly, let $D$ be an $m \times m$ diagonal matrix such that $d_{ii} = k_i$. Then it holds:
    $$\bar{L} = Q \cdot D \cdot Q ^T$$
\end{lemma}

Then the prove follows immediately using the Cauchy-Binet formula.
We similarly remove one row from the Q matrix (and one column from the transposed one), then the determinant of the laplacian for a multigraph without a row or a column equals
$$\det(\bar{L'}) = \det\left((Q')\cdot(Q')^T\right) = \sum_{S \in {n-1 \choose [m]}} \det\left(Q_S'\right) \cdot \det((Q^{S})') \cdot \prod_{i \in S} k_i$$
which is exactly the number of spanning trees in the simple graph using $S$ edges times the multiplicity of all those edges.

\end{document}
