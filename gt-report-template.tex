\documentclass[a4paper, 10pt]{article}
\usepackage[utf8]{inputenc} % Change according your file encoding
\usepackage{graphicx}
\usepackage{url}
\usepackage{amsmath}
\usepackage{amsthm}
\usepackage[a4paper, left=2cm, right=2cm, top=2cm, bottom=2cm]{geometry}

\newtheorem{obs}{Observation}
\newtheorem{theorem}{Theorem}

\theoremstyle{definition} % amsthm only
\newtheorem{definition}{Definition}

%\newtheorem{def}{Definition}

\begin{document}

Carlos Segarra \hfill Thursday, September 26th

\vspace{15pt}

\textbf{\Large Graph Theory: Matchings - Problem 1}

\vspace{20pt}

\textbf{\textit{(1) Prove that a $(2k+1)$-regular graph $\mathcal{G}$ with edge-connectivity $\lambda(G) \geq 2k$ has a perfect matching}}

\vspace{3pt}

First we will start with the definition of edge-connectivity and a reminder of a theorem (Tutte) we will use to prove the statement.

\begin{definition}[Edge Connectivity]
    Given a graph $G$, it's edge connectivity, $\lambda(G)$ is the minimum number of edges whose deletion disconnect G.
\end{definition}

From the definition, it is clear that $\lambda(G) \leq \delta(G)$, which in our case implies $2k \leq \lambda(G) \leq 2k + 1$. To see the former, if we remove all the incident vertices to the vertex of smallest degree, we disconnect $G$, proving to be an upper bound.

\begin{theorem}[Tutte]
    Let $G = (V, E)$ be a graph. For every $S \subset V$ let $c_o(G - S)$ denote the number of odd components of $G[V-S]$. The graph $G$ contains a perfect matching if and only if, for every $S \in V(G)$,
    $$c_o(G-S) \leq |S| $$
\end{theorem}
we denote the latter condition as the Tutte condition.

Further, let us make the following observation on $G$'s cardinality, where we use the Handshaking lemma and the fact that $G$ is $(2k + 1)$-regular.
$$2|E(G)| = \sum_{v \in V(G)} d(v) = |V(G)|(2k + 1) \Rightarrow |V(G)| \equiv 0 \mod 2$$

Knowing that the cardinality is even, lets us easily check that Tutte's condition holds if $|S| = 0$. From now on we will then assume $|S| > 0$. Again, using $G$'s regularity, the following inequality holds:
\begin{equation}\label{edge-ineq}
\#\lbrace\text{edges incident to S from odd components}\rbrace \leq \#\lbrace\text{edges incident to S}\rbrace \leq |S|(2k + 1)
\end{equation}

Let now $C$ be an odd component of $G - S$. If we count all the vertices incident to some vertex of $C$, we have from one side the $(2k + 1)$ regularity, and from the other the edges within $C$ (beginning and end inside C) and those leaving $C$. By construction, all edges leaving C must necessarily be incident to $S$ (see the supporting drawing). Hence, the following holds:
$$|C|(2k + 1) = 2|E(C)| + |E(C,G-C)| = 2|E(C)| + |E(C,S)|$$

Moreover, from the previous equality we observe that LHS has the parity of $|C|$, and the RHS has the parity of $|E(C,S)|$. Thus, $|C|$ and $|E(C,S)|$ have the same parity. In particular, $|E(C,S)|$ is odd. Lastly, given that $S$ disconnects $G$, by definition of edge connectivity we have
$$|E(C,S)| \geq \lambda(G) = \lbrace 2k, 2k + 1 \rbrace \overset{|E(C,S)| odd}{\Longrightarrow} |E(C,S)| \geq 2k + 1$$

Inserting this last inequality to (\ref{edge-ineq}) we have:
$$c_o(G-S)(2k + 1) \leq \#\lbrace\text{edges incident to S from odd components}\rbrace \leq |S|(2k + 1) \Rightarrow c_o(G-S) \leq |S|$$
which is exactly Tutte's condition, and hecne $G$ contains a perfect matching.

\end{document}
